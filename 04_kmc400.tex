\label{chap_kmc400}

The Kangaroo foundation has contributed significantly to the development of this project. The experts that work with the foundation have provided us information on their research methodologies. They also provided us two very interesting data-sets and were willing to test all our prototypes. In return we helped them analyze and make sense of these complex data-sets. This collaboration enriched the project and allowed the experts in the foundation to extract information from the data efficiently. This chapter will illustrate the benefits of this collaboration and show how the Braviz platform can provide value in a real study. 

\section{Background}


- What is KMC

%- The first randomized trial
Between 1993 and 1996 a randomized controlled trial comparing short and middle term outcomes of KMC in comparison to traditional treatment. The sample consisted of 746 infants born with low weight (less than 2000g.) at \emph{Clinica San Pedro Claver} in Bogotá \autocite{charpak_current_1996,charpak_kangaroo_1997,charpak_randomized_2001,charpak_kangaroo_2005}. Elligible infants were randomly assigned to the KMC or traditional intervention. KMC group was on kangaroo position 24 hours a day and fed by breast feeding with an optional supplement (preterm infant formula) derived by spoon or dropper in order to guarantee a weight gain of 15 grams per kilogram per day until the reach of term. Children in the traditional group were kept in an incubator until they were able to control their temperature and are gaining weight at an acceptable rate. Infants were discharged according to standard hospital practice. Both groups were followed until one year of corrected age. 

%-- What were the main findings

During this year there were fewer deaths in the KMC group as well as fewer visits to the hospital caused by infections. In addition mothers from the KMC group felt more competent. Griffiths and INFANIB tests showed higher development in the KMC group. The effect was more prominent on those born before 32 weeks of gestational age and those who went trough ICU. There was also a high impact of KMC on the development of personal relations and on planning functions related to brain development \autocite{tessier_kangaroo_2003}.

%\autocite{charpak_kangaroo_1997}
%\autocite{charpak_kangaroo_2005}
%\autocite{charpak_randomized_2001}
%\autocite{charpak_current_1996}

%(preguntar a Rejean y Nathalie)
-- Findings from other studies
-- KMC in the world

\section{Pilot Study}

In 2009, thirty-nine children who were part of the original study were relocated. The sample was composed of children born at or before 33 weeks of gestational age, 21 belonged to the KMC group and 18 from the traditional group. In addition, 9 kids born at term at the same hospital were recruited. These kids went trough

What tests



The pilot study
 - Data
 - TMS findings
 - IMAGINE


\section{Saving Brains Project}

%The saving brains project

Between 2013 and 2014, with financing from Canada Grand Challenges, a sample of X subjects from the original study was retrieved. By this time the participants were between 18 and 20 years old. The study aimed at identifying if the KMC intervention had a long term effect protecting the brain against cognitive, social, and academic difficulties. Specifically, by analyzing data about  brain maturation and structure, mental functions and behavioural patterns exhibited in family, school and work. For this purpose the following data was collected

% - Data dimensions
% -- Clinical
% -- Psychology
% -- TMS
% -- MRI




 - Objectives
 - Data Collection
 -- Process
 --- how?
 --- when?
 -- Participants

 
 - Data Processing
 -- Pipeline
 -- Quality Control
 --- Early assessment
 --- Processing errors
 --- Pathologic Data
 -- Bundles Statistics
 -- Geometric descriptors

 - Exploratory Analysis
 -- Use cases with Nathalie
 -- Examples of obvious and non obvious findings



