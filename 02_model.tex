%Introduction

The human brain can be analyzed from several different points of views. Some disciplines look at it from the outside, and try to understand how it responds to stimuli, how it behaves on different conditions, how it adapts to new contexts and how it changes over time. Some disciplines look at it at cellular and molecular level, trying to understand the chemical reactions that go on inside each cell making it work. Others analyze the activity of a groups of cells analyzing how they cooperate. Some focus on larger structures composed of several cells, trying to understand how they connect to each other and how the different types of organs complement each other. It is of special interest seeing how it matures over time, and how it recovers from traumatic events. Additionally it is important to analyze how it degenerates over time, and what diseases can affect it in order to create treatments and help the brain heal. 
These tasks are carried out by different specialists in very different environments. There are also a wide range of tools designed for studying the brain, from microscopes and voltage clamping techniques, to psychology instruments. There are also studies based on animals with similar structures or even electronics circuits specifically built to emulate the human brain. 

This shows that understanding the brain is far from easy, and that it involves an enormous amount of skills, tools and knowledge. It can also be seen that all of these information gathered from different perspectives must be integrated in other to get a full understanding. This will require teams from diverse specialties and contexts to work together, each providing one piece of the puzzle. 

This task will also require support from computational tools in order to be efficient. These tools should allow experts from different contexts to be productive as teams and to integrate data acquired using a wide array of methods. Data itself will be highly heterogeneous and there will be lots of it. As tools advance we become more efficient at making experiments and generating data, and the bottleneck has become analyzing it. In other words, data is acquired at a fastest rate than it can be understood. 

This is indeed a complex scenario, and it is evident that a single software tool will not be able to solve all the problems. Tackling the whole problem at once is also an impossible tasks. We need to break down the problem into simpler tasks that can be attacked without loosing the big picture. Doing this analysis is itself a challenging task, that can't be addressed lightly. Fortunately this kind of problems are found in several business applications and  software engineering techniques that can manage them have been developed. 

As mentioned previously, the problem of integrating brain data involves several points of view and several types of work-flows. This is a clear signal of the need for different applications instead of a single, all-mighty application. Nevertheless these applications will likely share several aspects. Analyzing these commonalities and differences is the first step towards a solution to the problem. Methodologies to do this can be borrowed from software product line engineering, model based software engineering and generative programming. The first task that needs to be done is collecting and organizing as much as information as we can on the domain, which will be the basis of a domain model. At this step the scope of the domain will also be accurately selected and described. Afterwards the commonalities and differences of the needs of the different stake-holders, work-flows, and data in the scoped domain will be compiled into a feature model. This model will be the basis for a proposal of an applications family, where each application addresses a particular problem inside the domain. 


\section{Domain Analysis}

\subsection{Definition}

Domain Model

- Stakeholders
- Users
-- points of view
-- activities
-- teams
-- selfish

-- Alternatives
-- choose
- Scope
-- Alternatives
-- Choose
- Define domain
-- Examples
-- Main Features
-- Relationship to other domains

\subsection{Characterization}

- Analysis of existing Applications
-- Commonalities and Differences
-- From other domains, need to be compatible
- Analysis of literatures and experts
- Domain Terms
- Domain Concepts
- Variability of domain concepts
-- feature model in problem space?

\section{Solution Proposal}


\subsection{Platform}

Proposal for Braviz platform

- Data Flow
- Task 
- Use Case
- Deployment


\subsection{Applications Family}

Proposal for Braviz application family

- Feature model
-- problem space
-- solution space

- rationale for each choice
- when to select each feature
- how to choose between alternatives
