%Introduction

The human brain can be analyzed from several different points of views. Some disciplines look at it from the outside, and try to understand how it responds to stimuli, how it behaves on different conditions, how it adapts to new contexts and how it changes over time. Some disciplines look at it at cellular and molecular level, trying to understand the chemical reactions that go on inside each cell making it work. Others analyze the activity of a groups of cells analyzing how they cooperate. Some focus on larger structures composed of several cells, trying to understand how they connect to each other and how the different types of organs complement each other. It is of special interest seeing how it matures over time, and how it recovers from traumatic events. Additionally it is important to analyze how it degenerates over time, and what diseases can affect it in order to create treatments and help the brain heal. 
These tasks are carried out by different specialists in very different environments. There are also a wide range of tools designed for studying the brain, from microscopes and voltage clamping techniques, to psychology instruments. There are also studies based on animals with similar structures or even electronics circuits specifically built to emulate the human brain. 

This shows that understanding the brain is far from easy, and that it involves an enormous amount of skills, tools and knowledge. It can also be seen that all of these information gathered from different perspectives must be integrated in other to get a full understanding. This will require teams from diverse specialties and contexts to work together, each providing one piece of the puzzle. 

This task will also require support from computational tools in order to be efficient. These tools should allow experts from different contexts to be productive as teams and to integrate data acquired using a wide array of methods. Data itself will be highly heterogeneous and there will be lots of it. As tools advance we become more efficient at making experiments and generating data, and the bottleneck has become analyzing it. In other words, data is acquired at a fastest rate than it can be understood. 

This is indeed a complex scenario, and it is evident that a single software tool will not be able to solve all the problems. Tackling the whole problem at once is also an impossible tasks. We need to break down the problem into simpler tasks that can be attacked without loosing the big picture. Doing this analysis is itself a challenging task, that can't be addressed lightly. Fortunately this kind of problems are found in several business applications and  software engineering techniques that can manage them have been developed. 

As mentioned previously, the problem of integrating brain data involves several points of view and several types of work-flows. This is a clear signal of the need for different applications instead of a single, all-mighty application. Nevertheless these applications will likely share several aspects. Analyzing these commonalities and differences is the first step towards a solution to the problem. 
Methodologies to do this can be borrowed from software product line engineering \autocite{pohl_software_2005}, model driven software engineering \autocite{brambilla_model-driven_2012} and generative programming \autocite{czarnecki_generative_2000}. 

The first task will be the scoping of the domain and the selection of a domain where a it is feasible to make a contribution. Afterwards the commonalities and differences in the needs of the different stake-holders, work-flows, and data in the selected domain are analyzed in order to create a feature model. This model will be the basis for a proposal of an applications family, where each application addresses a particular problem inside the domain. 


\section{Domain Engineering}

Domain Engineering is the practice of selecting and characterizing the domain in which the application family will be built. A good description of the steps required for this task can be found in \autocite{czarnecki_generative_2000}. We will follow these steps in order to identify the domain where we will work on the rest of the thesis.

\subsection{Domain Scoping}

As noted previously brain research involves several specialties, skills and techniques. It could be argued that because the brain is involved in almost every human action, all human and social sciences are at the end studying the brain. However in this project we will focus on more direct studies of the human brain. In particular we want to analyze its physical structure. Under this condition there are still a broad ways of looking at it.

In domain engineering the decision of where to focus must be also influenced by the strengths and experience of the organization, in this case our research group. Previous projects of the group (\footnote{Proyectos de Darwin, Jaime, Marcela}) have principally focused on analysis of medical images at m.m. scale, acquired by CT and MRI machines. We have good relationships with radiology departments at several hospitals and therefore access to images and, more critical, domain experts. MRI is an specially interesting technique as it does not produce ionizing radiation, and therefore is harmless for the subject. While CT and general x-rays involve radiating the human body. This is not harmful at small doses but the effects may accumulate over time. Therefore these techniques must be used with care and only when there is a valid medical reason that justifies it. On the other side, MRI scans can be applied to any subject and there is no need for a medical justification (but probably the study must be approved by an ethical committee). MRI scanners are also versatile machines which can acquire numerous kinds of images. Structural images can be acquired at different configurations which provides better contrast for different tissues or molecules. Advanced techniques grant the ability to do spectroscopy in order to characterize the composition of specific areas of the brain. By using contrast agents it is also possible to precisely locate specific proteins, cells or structures. 
For these reasons we chose to focus our development on images acquired by MRI. We also chose to focus on T1 and T2 weighted structural images, Diffusion weighted images and BOLD f-MRI. This decision was also caused by the previous experiences in the group, but nevertheless keep in mind that domain definition and scoping is an iterative process, and therefore this decision will certainly be revisited in the future.

An objective of the project from the start has been the integration of data, therefore even though we are focusing on MRI data, we need additional data to provide context and therefore a more complete picture. Recall that one of our hypotheses is that the brain is better understood by teams of specialists who can bring different perspectives. One of our challenges was linking structure and function of the brain. This function of the brain may signify quick reactions to stimuli, like for example catching a ball, all the way to complex social behaviors over several years. This kind of data can be collected by economists, psychologists, epidemiologists, and sociologists among others. This information can be very complex, but a non trivial subset of it consists of numerical, nominal and ordinal variables which can be registered in spreadsheet tables. We will attempt to integrate data from these diverse set of disciplines if it is presented in a table-like format, but we will not try to interpret the data in any way. This responsibility will fall on end users. 

Finally there is another important kind of data we want to consider as it provides a perfect link between structure and functioning of the brain. This is, data from TMS exams. We are very lucky to have access to neurophysiologists specialized in this kind of exams, which can further enlight the functioning of the brain and its relationship to its structure. To recapitulate, the current project is going to consider the following kinds of data:

\begin{itemize}
\item MRI brain images
\begin{itemize}
\item Structural T1 and T2 weighted
\item Diffusion Weighted Images
\item Functional MRI
\end{itemize}
\item TMS exams
\item Tabular data from other disciplines
\begin{itemize}
\item Nominal variables
\item Ordinal variables
\item Numeric variables
\end{itemize}
\end{itemize}  



Nevertheless it is worth reiterating that domain scoping is an iterative process, which has to go on for the duration of the project. 

Domain Model

- Stakeholders
- Users
-- points of view
-- activities
-- teams
-- selfish

-- Alternatives
-- choose
- Scope
-- Alternatives
-- Choose
- Define domain
-- Examples
-- Main Features
-- Relationship to other domains

\subsection{Domain Modeling}

- Analysis of existing Applications
-- Commonalities and Differences
-- From other domains, need to be compatible
- Analysis of literatures and experts
- Domain Terms
- Domain Concepts
- Variability of domain concepts
-- feature model in problem space?

\section{Solution Proposal}


\subsection{Platform}

Proposal for Braviz platform

- Data Flow
- Task 
- Use Case
- Deployment


\subsection{Applications Family}

Proposal for Braviz application family

- Feature model
-- problem space
-- solution space

- rationale for each choice
- when to select each feature
- how to choose between alternatives
