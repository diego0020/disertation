
The ideas reflected in the previous chapters were implemented into a software platform called Braviz. This software is licensed under a LGPL license and its source code and documentation can be found online at http://diego0020.github.io/braviz . As suggested by user centered design, the software was the result of several iterations. The architecture of the software reflects the model described in chapter \ref{chap_model}. Details of these aspects will be given in the following. Finally we will describe some of the more technical details of the software.

\section{Iterations}


%Iterations
%-------------

%- Feedback
%- changes
%- evidence of user centered design

The development of the platform went trough several prototypes that were tested with target users. From each prototype several lessons were learned, both from the technical point and from the users feedback. Each iteration brings new ideas and challenges to the project. As the process goes on prototypes become more sophisticated. At the start changes occur very fast, while at the process goes on changes are smaller and take longer. The whole process can be seen as a spiral (see figure \ref{intro_spiral}). This section will provide an overview of each of the iterations up to the current point, and show the more significant insights and changes from each of them. Notice that the process has not stopped, and we expect the platform to continue evolving. Chapter \ref{chap_conclusions} will provide details of the plans for future iterations.

\subsection{Previous Work}

\begin{figure}
\centering
\includegraphics[width=0.9\textwidth]{historic/kab_figura1.png} 
\caption{\label{fig_kab}The main interface of KAB}
\end{figure}

The first prototype built by our group was called KAB \autocite{castro_kab:_2012}. This tool was built to analyze data from the KMC pilot study \autocite{schneider_cerebral_2012}. This software integrated data from Diffusion MRI, Functional MRI and structural MRI. The main interface of the tool is shown in figure \ref{fig_kab}. Data was pre-processed using FSL \autocite{jenkinson_fsl_2012} for skull removal and FMRI modeling, and Medinria \autocite{toussaint_medinria} for diffusion data. Registration between diffusion and structural spaces was done on-line using the tool itself. The most important features of the software were:

\begin{itemize}
\item Visualizing structural MRI, fMRI and tractography on the same space
\item Selection of bundles in a full tractography
\begin{itemize}
\item Using two spherical regions of interests located manually
\item Using conical regions of interests, representing the spread of TMS magnetic pulses
\item Using areas where fMRI statistics were above a threshold
\item Using hand-drawn regions
\end{itemize}
\item Generating statistics from selected bundles
\item Automatic generation of reports
\end{itemize}

Integrating information from different sources was the major contribution of the tool. Traditionally each type of data was analyzed on its own, using dedicated tools, and integrating information in this way was beyond brain experts. This tool was implemented based on the BBTK framework \autocite{hoyos_creatools} developed by Creatis. 

Specialists appreciated the ability to integrate several kinds of data and ask questions that involved relationships between them. During the development of the application specialists started asking questions about symmetries in the brain, and therefore the option of reflecting a ROI to the other hemisphere was added. Specialists also expressed the need of linking the coordinate system of the tool to a known atlas.

While this tool made evident the interest in integrated analysis of data, it also had some limitations.
\begin{itemize}
\item It required too much manual work, often repeating tasks
\item It had too many features in a single application, which made it complex
\item Manual registration was very error prone
\item It didn't integrate non-image data.
\end{itemize}
Based on these we proposed a new set of prototypes.

\subsection{First prototypes}

%-- titanic

For the first round of prototypes after KAB, we focused on creating tools with a limited set of features, which could run faster and be easier to learn. We stayed with the BBTK framework as it provided us a fast way to iterate. At this stage our main objectives were solving the technical problems we identified in KAB. Specifically
\begin{itemize}
\item Perform registration automatically
\item Integrate non-image data
\item Increase computational performance
\end{itemize}

\begin{figure}
\centering
\includegraphics[width=0.9\textwidth]{historic/Titanic_B.png} 
\caption{\label{fig_titanic}Prototype integrating tabular and image data}
\end{figure}

Most of the prototypes created at this stage were only used internally to test different algorithms and techniques. Figure \ref{fig_titanic} shows the final prototype of this series. Its main features are
\begin{itemize}
\item Display values of clinical variables together with the image
\item Explicit identification of the current subject
\item Simple interface
\item Selection of different image modalities
\item Selection of different coloring schemes for bundles
\item Direct selection of the most important fiber groups
\item Control on most important visualization parameters
\end{itemize}

\begin{figure}
\centering
\includegraphics[width=0.9\textwidth]{historic/Titanic.png} 
\caption{\label{fig_titanic_2}The prototype could be configured to show several images and to color fibers in different ways.}
\end{figure}

Figure \ref{fig_titanic_2} shows several of the views that could be accomplished with this tool.
The tool was simple to use as there were few controls and all of them were visible. Notice this application was tailored towards the specific project. At the time the research team was specially interested in corpus callosum and motor fibers. This application allowed researchers to quickly answer specific questions relating fiber bundles to clinical variables. Also now researchers from all specialists could look at tractography and a different MRI images painlessly, where before they had to rely on large, hard to use applications. Data was organized in drive in a well defined structure. When the application launches the user only needed to select the main structural image, and the rest of the necessary files were located trough their relative paths. There was no need for end user to even know about the existence of transformation matrices or other files. This simplified a lot the process of simply looking at spatial data in context, and allowed experts to do this in a convenient way. Having a more direct access to spatial data allowed experts from different specialties to look at them on their own time, and therefore come up with new ideas for analyzes. They were also using the tool to communicate ideas to other brain experts and also to us. We now had a better way of communicating with experts, and therefore we could get a better understanding of their needs. This tool showed us the potential of integrating spatial data with clinical data, but some limitations were made evident in our work with experts
\begin{itemize}
\item It was not possible to change subject during an analysis
\item It was not easy to select different variables
\item It was not possible to use different sets of tracks
\item It is cumbersome to compare different subjects
\end{itemize}

Analyzing groups of subjects and comparing subjects became the priority for the next round.

\subsection{Braviz, First Version}

%- Braviz/tk
%-- sample applications

For the next cycle we dropped the BBTK platform as we found out it was limiting too much our design space. The platform enforced an execution model that kept everything updated. This was great for small applications, that were meant to run for just a couple of minutes. However for large applications that could run for hours we required more control of when the different modules would execute. BBTK relied on VTK for all visualization tasks, and therefor we decided to start developing directly on VTK. However it was still very important to us to have quick development cycles, and we were willing to sacrifice some performance to get it, therefore we switched our main developing language to python.

\begin{figure}
\centering
\includegraphics[width=0.9\textwidth]{historic/surf1.png} 
\caption{\label{fig_surf_1}A freesurfer surface inside Braviz}
\end{figure}

\begin{figure}
\centering
\includegraphics[width=0.9\textwidth]{historic/grid_and_scatter.png} 
\caption{\label{fig_grid}Corpus Callosums of several subjects organized on a grid and colored with respect to clinical variables.
A scatter plot of these variables is shown at the bottom left}
\end{figure}

\begin{figure}
\centering
\includegraphics[width=0.9\textwidth]{historic/fibers_in_context.png} 
\caption{\label{fig_fibers_ctx}Fibers could be selected using freesurfer segmentation. Vtk widgets could be used to make measurements.
Additionally statistics of each bundle were calculated and shown instantaneously.}
\end{figure}

\begin{figure}
\centering
\includegraphics[width=0.9\textwidth]{historic/rejean2.png} 
\caption{\label{fig_fmri_1}This prototypes make the time aspect of fMRI explicit, therefore helping non-experts understand its meaning and make better interpretations}
\end{figure}

\begin{figure}
\centering
\includegraphics[width=0.9\textwidth]{historic/braviz_menu.png} 
\caption{\label{fig_menu_1}A menu was provided to give an overview of the available applications and provide simple access.}
\end{figure}

\begin{figure}
\centering
\includegraphics[width=0.9\textwidth]{historic/mini_tms.png} 
\caption{\label{fig_tms_1}An application tailored at analyzing TMS variables relationship with tractography}
\end{figure}

\begin{figure}
\centering
\includegraphics[width=0.9\textwidth]{historic/cc_in_context.png} 
\caption{\label{fig_cc_ctx_1}Fibers of the corpus callosum with segmentation and anatomical image as context}
\end{figure}

\begin{figure}
\centering
\includegraphics[width=0.9\textwidth]{historic/compare.png} 
\caption{\label{fig_compare_1}Comparing the motor tract of two subjects}
\end{figure}

\begin{figure}
\centering
\includegraphics[width=0.9\textwidth]{historic/mini_star.png} 
\caption{\label{fig_star_1}This application allowed comparing several structures and variables between groups of subjects.}
\end{figure}

\subsection{Braviz, Second Version}

%- Braviz Qt
%-- sample applications


\section{Architecture}

From model to implementation
-----------------------------

- architecture
- common platform
- skeleton of a application
- database


\section{Technical Details}

Technical details
------------------

- organization
- libraries
- messaging protocol
- data input/output
- transformations
- coordinate systems



