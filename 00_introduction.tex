

One of the challenges in brain research is finding relationships between the physical structure of the brain and the way it functions. Structural information is gathered mainly through the use of imaging techniques as Magnetic Resonance Imaging (MRI), Computer Aided Tomography (CAT) or Positron Emission Tomography (PET). Other methods measure the electrical activity of the brain, as in Electro-Encephalography  (EEG) or Magneto-Encephalography (MEG). Transcranial Magnetic Stimulation (TMS) is a technique where cells are stimulated using a rapid changing magnetic field, which in turn generates an electrical field inside the neurons, the effects of this stimulation are afterwards measured elsewhere. 

Traditionally research is done by formulating hypotheses and designing experiments to test such hypotheses. Next, recruiting subjects, performing the experiment on each of them and gathering data. Finally this data would be analyzes using statistical methods which provide evidence in favor or against the hypothesis. This methodology imposes limitations on how data is used. Usually data is only used once, which is a shame because gathering this data is expensive in time, effort and resources. 

In the past years there has been a shift towards gathering data in a more open fashion, and several public databases have appeared. There has been several improvements in the way data is collected, stored and shared, both at the technical level and at the policies level. Even inside small research groups, it has become usual to keep looking at the data after traditional hypothesis testing is complete. 

All of this is leading to a change in the way research is done. This is a shift from hypothesis driven research into data driven research. This also creates an increased need for exploratory analysis methods, where the ecosystem is dominated by methods created for confirmatory analysis. In this context new challenges appear. It is now necessary to manage, analyze and visualize data where the number of subjects is increased by orders of magnitude, as well as the measures available for each one. In this scenario it is hard to guarantee homogeneity on the data belonging to each subject. In several cases there are measures available for each subject at several points in time. 

The work-flow in this kind of analysis differs significantly from the traditional one. It requires iterating trough the data several times, looking at it from different points of view, searching for relevant subjects and measures, gathering details from individuals and performing group analyzes involving several measures. 

Similar scenarios have appeared in other domains, as in economy, terrorism prevention and business intelligence. The challenge is always extracting meaningful information from large and heterogeneous data-sets. Approximations to the problem often involve statistics, machine learning and databases together with efficient and intuitive interfaces and data visualizations. Visual Analytics has emerged as a discipline which attempts to integrate all of these areas with the objective of making an optimal use of the available data. Visual Analytics recognizes the human analyst as the most important element in the task, and focuses on letting the analyst work with freedom and efficiency and focused as much as possible on the data instead of the tools.

In this thesis visual analytics techniques are applied to the particular case of cohort studies in brain data. A model which abstracts and formalizes the elements of this task is proposed. This model can be adapted and applied to other domain where cohort-like data is found. The model is materialized in a software environment called BRAVIZ. This software was successfully used in a large brain study performed by the Kangaroo Foundation. The results are very encouraging and show that the proposed tools really helped the experts feel like they really owned their data and could move trough it freely and explore it as they liked. This created a pleasant experience where information, questions and hypotheses could be gathered from the data.


\section{Visual Analytics}

Economia, como freakanomics, datos de censos, datos publicos.

\section{Exploratory and Confirmatory}

Data Driven Research

\section{Exploratory in Brain Research}

\section{User Centered Design}
