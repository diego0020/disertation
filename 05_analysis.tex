%\section{Retrospective}

%¿cómo estabamos al principio?

When this project started we had access to numerous image analysis and visualization tools (see chapter \ref{chap_related}). However these tools are designed to work with specific data types, and thus integrating data from different sources required significant work. Additionally most of these tools have a step learning curve, which is appropriate for image experts but leaves them out of reach for users from other domains. Experts from domains different to radiology required to go trough the radiologist or an engineer in order to access image data. Moreover clinical data was always on separate files, which meant it was necessary to switch application in order to get context information about a subject. Typical image based analysis involved extracting scalar features from images, which were later added to a table, and then analyzed in statistical software together with clinical data. This analysis was usually completed using tables. Generating plots required additional steps and was done only for communication purposes. The risk of this approach was that outliers or pathological data could go undetected. 

The current project was conceived to address these issues. Using the proposed tools it is possible to access image data instantaneously,  for image experts, but also for experts from different domains. In addition these images are showed with context and they can be linked to additional information of a given subject. Scalar features extracted from image data can be visually analyzed together with clinical data. In this way outliers immediately draw attention, can be identified and analyzed in more detail. Experts from all domains can access spatial data, think about it, and propose analyzes. In contrast, before most analyzes involving spatial data were proposed by image experts and limited to existing tools. These analyzes will still need the help of image experts for performing the required calculations and interpreting the results; and possibly from engineers if new tools need to be developed. By making data available to the whole team, this way of thinking becomes possible.    

% ¿Como ha anvanzado nuestra comprension del problema?
% ¿de los datos?
% ¿de los usuarios?
% ¿de las necesidades?
In order to better assess the benefits and limitations of the proposed approach, we conducted several interviews with potential users, applied surveys, and did a group session. These activities and their results will be explained in this chapter. Additionally it will be explained how the model from Chapter \ref{chap_model} can be used to generate applications requested by users.	

% ¿qué problemas nuevos han surgido?
% ¿qué oportunidades?
% ¿hacia dónde vamos?

\section{Collaboration with Team Cave}

\footnote{A better name is needed, help Cyril} An important part of the requirements analysis for the project was carried out at CHUL in Quebec, Canada. Specifically, the Axes Neurosciences group was a significant contributor. This group researches the functioning of the nervous system in healthy and pathological conditions. One of the methods used is performing experiments on large numbers of participants. During these experiments they collect data from each participant using neuro-psychological tests, measurements of muscle strength and movement, neuro-image and TMS tests. This center is located inside the Laval Universiy Hospital (CHUL) in Quebec.

Single or double TMS was being used for measuring the quality of nerve fibers, and repetitive TMS was being used as therapy on muscles \footnote{Maybe cite Veronique's paper as an example}. Data from TMS experiments was captured using EEG equipment and the \emph{Lab Chart} software. These signals were then post processed in the same software and scalar measures were extracted. These measurements were then copied to an excel table where they were combined with subject data and other collected variables; for example the feeling of pain, the strength of a muscle, or the time taken for completing a nine hole peg test. Afterward the excel table would be re-organized and fed into statistical software. The main statistical software used at this stage were \emph{Statistica} and \emph{Prism}. Recently the lab acquired a license for \emph{Aabel}, which some researchers like better as it provides very good explanations of the meaning and methods of each analysis. Also the manual is rich with statistical theory and examples. The tool also provides interactive graphs which can be used for exploratory analysis. Notice that some of the software is proprietary and there were limited licenses, therefore it was common for researchers to have to move data trough different machines using usb drives. Internet was not an option as the hospital network is very restrictive.

%Description of the lab, software used, prism, aabel, statistica, labchart, excel, 

Can I add some pictures from Cyril lab ?

% Knwoldedge depends on the hypotheses, and the hypotheses on the technology available for testing the questions raised. 
 
% Cyril's team
% Cyril's TMS publication

%The tms applications

One of the interests of the group was analyzing relationships between TMS outcomes and white matter structure as portrayed by DWI and tractography. The particular TMS experiment tested the connections from the primary motor cortex to the hand muscles, and between the motor cortices at both hemipsheres (see \autocite{schneider_cerebral_2012}). For this purpose an specific application was designed, it showed the values of the TMS outcome together with visualizations of the relevant fiber bundles. Screen-shots of the finished application can be seen on figures \ref{fig_tms_1} and \ref{fig_tms_2}). 

\begin{figure}
	\centering
		\includegraphics[width=0.90\textwidth]{figures/analysis/tms_view_early}
	\caption{An early version of the TMS viewer application.}
	\label{fig_tms_view_early}
\end{figure}

Figure \ref{fig_tms_view_early} shows an earlier version of the application. In the top left side it had a combo-box for selecting one of the different outcomes. The radio buttons below allowed the researcher to choose between the metrics for the dominant or non-dominant side. Finally there was a list of subjects which could be filtered to show only males, only females or both. At the top right there was a 3D viewer which would display a relevant fiber bundle related to the current outcome. Specifically for intra-cortical measures it would show the motor fibers for the corresponding hemisphere (as in figure \ref{fig_tms_1}) while the corpus callosum would appear for Inter-hemispheric measures.
Below there was a bar plot showing the values of the current outcome for the current subject, together with a large display of this value on the right. The solid and dotted lines in the plot correspond to the mean and the interval within one standard deviation; these statistics are calculated considering only the subjects in the control group. Bars inside this interval are colored green while those outside are colored in red.
This layout was first designed on paper with one of the group researchers. 

The user could move trough subjects either by selecting a code on the list or by clicking on one of the bars. The bar at the right would change to the new value, using a transition to accentuate the difference. The image will also change to that of the new subject. This simple tool allows the researcher to navigate between the TMS-outcomes while looking at the underlying white matter structure. 

\begin{figure}
	\centering
		\includegraphics[width=0.90\textwidth]{figures/analysis/tms_view_motor}
	\caption{A second version of the TMS-viewer application}
	\label{fig_tms_view_second}
\end{figure}

Figure \ref{fig_tms_view_second} shows a posterior version of this tool. After receiving feedback from the group, several improvements were made:
\begin{itemize}
	\item TMS outcomes are now organized in a tree, and more descriptive names are used. The purpose is to provide meaning to experts who have limited experience with TMS. Additionally tooltips were added in order to provide additional information on each metric.
	\item The underlying study involved three groups: kangaroo mother care, incubator and control. For some analysis it is convenient to be blind about these groups, but some other times it is beneficial to have this information. Now there is an option to show the different groups in the list of subjects by using different background colors.
	\item The axes in the plot are more clearly labeled and measurement units are shown. 
	\item For some outcomes a higher value is better, while for some others a lower value is better. Now values of the second type are showed using squares instead of bars.
	\item Colors in the main plot were redefined. Now red can be read as \emph{worst than controls}, yellow as \emph{similar to controls} and green as \emph{better than controls}.
	\item The main plot can show the complete sample or only some interesting subjects. The user may add or remove subjects to this sub-sample using the buttons below the subject list.
\end{itemize}

Another round of feedback produced the version shown in figure \ref{fig_tms_2}. This version included additional information in the 3D viewer in order to help users interpret it. The other major feature was showing grouped data in the main plot for the three groups in the study.

This tool allowed experts to .................\footnote{Help Cyril}.


% Veronique Manip
% finding correlations

Another common task found in several of the current experiments was finding correlations between dependent variables. This process was usually done by testing each pair of variables at a time, but \emph{Aabel} provided a correlation matrix. This idea was nice, but it could be improved. In a group with several young researchers\footnote{Vero, Hugo, LD} an application for testing correlations was designed. The result of this effort is the \emph{Correlations} application shown in figure \ref{fig_correlations}. This application is now included in the Braviz set, but can also be used as a stand-alone application (available at \url{https://github.com/diego0020/correlation_viewer}). In the stand-alone mode the application loads data from an excel file where each row is a subject and each column a variable. The tool interface is divided in three. The first component is a list of variables, each of them with a check-box; it is used to add or remove variables from the analysis. Next is a correlation matrix which encodes in colors the degree of correlation between each pair of variables. And finally, when the user clicks on a square of the matrix the corresponding scatter plot is shown together with the $r$ and $p$ values from the Pearson correlation. An important feature is that individual points can be removed from analysis by clicking on them in the scatter plot. This feature was added because commonly there are extreme cases which give the impression of a high correlation, however this effect disappears when the point is removed. The application was refined trough several feedback sessions in the group, and several bugs were discovered and corrected.  This tool is now currently used in this lab\footnote{maybe ask hugo, v and ld from some experineces}.

The \emph{Anova} and \emph{Linear Model} applications (see figures \ref{fig_anova_2} and \ref{fig_lm_2}) were also inspired but the current research questions and workflows in this team. Most of the time they have a treatment variable, several confounders, and outcomes; and the objective is assessing if the treatment has any effects after controlling for confounders. This is usually done using ANOVA or two way ANOVA if before and after data is available. The two way ANOVA feature has not been implemented in the main Braviz tool, but is in the list of wanted features \footnote{what else can be said here?}. 

% Anna Belle Manip
% Hugo Manip
% 
% ANOVA - LM

% Alyssa

A pediatrician\footnote{Alyssa} who worked at the hospital provided her opinion of the system from the clinical point of view. She thought that such a tool would be very useful in order to talk the kid's parents, and better explain any findings. Also if the sample from the current study showed positive results in a similar case, this evidence could be shared with parents to reassure them. It would be also helpful to have all data from a given subject as well as reference data in order to understand the factors that affect a prognosis. 

%insights from cyril's interview
% Visuals are very important. 

In a follow up interview with the head of the lab (Cyril\footnote{Should Cyril be named explicitly?}) he gave us his impressions about the complete system. First of all he was concerned about how new users could get up to speed. He proposed that the first screen should guide the user. What should he do first? There should be a tutorial, and it should be enough so that the user can beyond intuitively. Additionally, he expressed that the user should never be puzzled by the tool itself. This means that the main format should be consistent. There should not be any surprises. The tool should also provide some guidance on relevant data for specific questions. 

As a possible application, he stated that the tool should help in understanding subjects which are not part of the original sample. Given a new subject, it should help me find similar subjects in the sample. It would be specially useful if there are no images of the new subject but the images of existing subjects with similar characteristics are available. 

He also emphasized that this system should be targeted to multiple users expert in multiple domains. The point was not transforming each user into an expert in everything, but to help experts work together and provide a full picture of each subject. With current technologies data from different domains are out of reach, because tools are not compatible, it's hard to understand, and the links between domains are not evident. This causes subjects to be divided into different dimensions. To counter this it is necessary to use a common language that makes links easier to see. By looking at each subject as a whole the tool humanizes research. In this way Braviz could also be very valuable in teaching. 

The tool should let users be very efficient at their work and never make them waste time. It should also be very accurate, as any mistake would cause experts to loose trust in the tool, and all benefits would be lost. No one would want to use a tool that produces results that must be double checked.

He agreed with the user centered approach, and remarked that tools should for end-users should be designed by those end-users, but engineers would always be required.

A final remark was that \emph{knowledge depends on the hypotheses, and the hypotheses on the technology available for testing the questions raised}. 


%%
%KMC
%	Marin
%	Rejean
%	Nathalie

% Video Conference from Iowa
% D:\dropbox\Dropbox\VaBD\ProyectoSavingBrains\minutas_demos_04_14.odt
% Ste Irenne

% lock patient

\section{Interest Surveys}
NIDCAP
Guttman's lab

In order to evaluate the interest of the community for a tool such as Braviz, a survey was administered at the NIDCAP\footnote{Por favor verificar con Nathalie} conference.  This conference was held in Barcelona on October 26, 2014. 
Assistants to the conference were ......................................\footnote{Help}. Seven experts answered.

Before the survey, Dr Charpak gave a demonstration of the tool. The first part of a question asked participants how much they agreed with a set of statements, using a likert scale. The results from this question can be seen in table \ref{tab_nidcap_likert}. 

\begin{table}
	\centering
		\begin{tabular}{p{0.6\textwidth}|ccccc}
			Statement&1&2&3&4&5 \\
			\hline
			The presented tools (Braviz) would support my line of work. & 5 &-&-& 2 &- \\
			Interactive visualization with access to individual detailed information is a key feature. &7&-&-&-&- \\			
			The possibility to explore images (MRI, DTI, fMRI) in the context of other variables, adds value to the analysis. &7&-&-&-&- \\
			The integration of basic statistical tools to the interactive exploration is a valuable feature &7&-&-&-&- \\
			I would be willing to test these tools using my own data in a near future (months) & 6 &-& 1 &-&- \\
		\end{tabular}
	\caption{Answers to the survey applied at NIDCAP, 2014. In the first part of the survey, participants should indicate how much they agreed with the listed statements. The scale was
	1: Totally agree, 2: Agree, 3: Neutral, 4: Disagree and 5: Totally disagree. The table shows the number of participants who answered with each value.}
	\label{tab_nidcap_likert}
\end{table}

The second question was: With what frequency do you make visualizations of your data? The possible answers and the number of participants who selected them were:

\begin{itemize}
	\item It's always the first thing I do : No one
	\item Almost always: four participants
	\item Only when I notice something odd: one participants
	\item Rarely: one participant
	\item Just to publish or socialize results: No one
\end{itemize}

The following questions asked : Which tools, similar to the one presented, do you know of? and how satisfied are you with these tools? The answers were 
\begin{itemize}
	\item Conventional T1, T2, MRI, DWI, FA. Very satisfied.
	\item Computarized order entry ( CPOE ). Very satisfied.
	\item Not a single tool, but several specialized tools: cognitive potentials, tractography, etc. Very satisfied.
\end{itemize}

The rest of participants left this space empty. The next question asked: What additional features would you like to find in a visual exploration tool? Participants wrote the following answers
\begin{itemize}
	\item Templates. Volume of structures and tractography data are best if used to combine with functional data in the follow up activity. 
	\item Vermis volumetric analysis.
	\item New techniques as analysis of electical activity using EEG and Bayley III curves for followup. 
	\item Orthogonal visualization of brains, ordinal variables, anova with nominal and ordinal outcomes, comparison with standard values.
\end{itemize}

This survey was also applied at the Center for Neurological Images at Brigham and Women's hospital in Botson, MA, USA. This activity was conducted on June, 2015. The methodology was also a quick demo followed by the survey. This group is composed of physicians (neurologists and radiologists), biologists and engineers, who work in brain image research. 

\begin{table}
	\centering
		\begin{tabular}{p{0.6\textwidth}|ccccc}
			Statement&1&2&3&4&5 \\
			\hline
			The presented tools (Braviz) would support my line of work. & 1 & 3 &-& - & - \\
			Interactive visualization with access to individual detailed information is a key feature. &4&-&-&-&- \\			
			The possibility to explore images (MRI, DTI, fMRI) in the context of other variables, adds value to the analysis. &4&-&-&-&- \\
			The integration of basic statistical tools to the interactive exploration is a valuable feature &4&-&-&-&- \\
			I would be willing to test these tools using my own data in a near future (months) & 3 & 1 & - &-&- \\
		\end{tabular}
	\caption{Answers to the survey applied at BWH, 20145. See table \ref{tab_nidcap_likert}.}
	\label{tab_bwh_likert}
\end{table}

The answers to the following questions were

\begin{itemize}
	\item With what frequency do you make visualizations of your data?
	\begin{itemize}
		\item It's always the first thing I do : one participant
		\item Almost always: two participants
		\item Rarely one participant
	\end{itemize}
	\item Which tools similar to the one presented, do you know of? How satisfied are you with these tools?
	\begin{itemize}
		\item Sear, very satisfied
		\item Spine, very satisfied
	\end{itemize}
	\item What additional features would you  like in a visual exploration tool?
	\begin{itemize}
		\item Interactive feature selection (for example all white matter lesions)
		\item Atlases, multiple statistics module (for both descriptive and analysis), database selection \& matching tools, export results and images, voxel-wise analysis, TBSS.
	\end{itemize}	
\end{itemize}

\smallskip

While the sample is small, the survey shows there are multiple specialists willing to try these tools. Most of the participants consider visualization an essential part of analysis, and find value in integrating spatial and clinical data. On the other hand none of them complained about the tools they currently use. This supports our hypothesis that Braviz will not replace any of these tools but complement them. Additionally the survey shows that there is a large interest in integrating more data types and statistical analyzes. Finally, several of the experts are willing to participate in longer studies, so in the future we will be able to collect more data about how these tools can be used in diverse projects.


\section{Group Test}

Ana Maria

\section{Building Applications}

Consequence of the model
Reusable components
Process, stages
metrics: time, lines of code
Other developers: David, Yoyis


\section{Discussion}

What features are the most useful?
What is feasible now that wasn't at the start?
What features are missing?
What limitations have come to light?

Can these techniques be used on other domains?



