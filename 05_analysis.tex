%\section{Retrospective}

%¿cómo estabamos al principio?

When this project started we had access to numerous image analysis and visualization tools (see chapter \ref{chap_related}). However these tools are designed to work with specific data types, and thus integrating data from different sources required significant work. Additionally most of these tools have a step learning curve, which is appropriate for image experts but leaves them out of reach for users from other domains. Experts from domains different to radiology required to go trough the radiologist or an engineer in order to access image data. Moreover clinical data was always on separate files, which meant it was necessary to switch application in order to get context information about a subject. Typical image based analysis involved extracting scalar features from images, which were later added to a table, and then analyzed in statistical software together with clinical data. This analysis was usually completed using tables. Generating plots required additional steps and was done only for communication purposes. The risk of this approach was that outliers or pathological data could go undetected. 

The current project was conceived to address these issues. Using the proposed tools it is possible to access image data instantaneously,  for image experts, but also for experts from different domains. In addition these images are showed with context and they can be linked to additional information of a given subject. Scalar features extracted from image data can be visually analyzed together with clinical data. In this way outliers immediately draw attention, can be identified and analyzed in more detail. Experts from all domains can access spatial data, think about it, and propose analyzes. In contrast, before most analyzes involving spatial data were proposed by image experts and limited to existing tools. These analyzes will still need the help of image experts for performing the required calculations and interpreting the results; and possibly from engineers if new tools need to be developed. By making data available to the whole team, this way of thinking becomes possible.    

%¿Como ha anvanzado nuestra comprension del problema?
%¿de los datos?
%¿de los usuarios?
%¿de las necesidades?
In order to better assess the benefits and limitations of the proposed approach, we conducted several interviews with potential users, applied surveys, and did a group session. These activities and their results will be explained in this chapter. 	

%¿qué problemas nuevos han surgido?
%¿qué oportunidades?
%¿hacia dónde vamos?

\section{Feedback from Users}

Cyril's team
Alyssa
Marin
Nathalie
Rejean

\section{Interest Surveys}
NIDCAP
Guttman's lab

\section{Formal Test}

Ana Maria

\section{Building Applications}

Consequence of the model
Reusable components
Process, stages
metrics: time, lines of code
Other developers: David, Yoyis


\section{Discussion}

What features are the most useful?
What is feasible now that wasn't at the start?
What features are missing?
What limitations have come to light?

Can these techniques be used on other domains?



