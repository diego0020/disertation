%Answer research questions, close the gap, close the introduction
The previous chapters presented a model that allows to design and implement software tools that support exploratory analysis of data from large brain studies with heterogeneous data from multiple subjects. This model was used to implement a set of applications that supported the analysis of data from a large brain study which included several types of data derived from neuro-images as well as clinical, demographic and socio-economic data. Further tests with potential users showed the relevance of this type of tools in today's research. Tools like the ones proposed in this work can change the way in which information is extracted from data and help in the transition towards a more collaborative and open research. 

\section{Effects in research workflow}

%- Explicit contributions to experts (neuroscience, images, research)

%- Subject as a whole
%-- Integrate data
%-- In context

One of the benefits of the proposed approach is that it lets all researchers look at data from different fields. The effect of this is that there is an integrated view of each subject. Data is not considered in isolation, but in the context of other subject data, as well as in the context of the sample. This additional context allows experts to make better interpretations of the data of focus. In addition, this allows researcher to raise questions and hypotheses relating the focus and context data. The benefits of integrating all study data in a single place were illustrated by researchers in the Kangaroo Mother Care study (see chapter \ref{chap_kmc400}) as well as physicians and researchers outside the foundation (see chapter \ref{chap_analysis}).

%- Deeper analysis
%-- Ask new questions
%-- complex questions
%- Extract more information from each data-set

The BRAVIZ system allows data-sets to be continuously enhanced by adding variables, annotations and structures. This increases the opportunities to ask and explore.  By using the proposed system researchers have gone beyond the original questions and hypotheses of the project and are proposing several new questions on the same data-set. 

Ideally, there would be a system where technical details never get in the way of the researcher's line of thought. In BRAVIZ, there are still moments where technical limitations get on the way of the analysis.  For example, the current implementation only permits spherical regions of interest (ROIs). These are very practical from the technical point of view, but impose limitation on the user. The system should continue improving in order to address these limitations.

%- Better understanding through visualizations
In addition to integrating data from multiple domains the system focuses on presenting data visually. As evidenced in the visual analytics' literature, rich interactive graphics are the most efficient way of transferring information from a computer to the expert. In the domain of this study the benefits of visualizations over tables of numbers are evident. Plots can provide more detailed descriptions of relationships, and they cause unusual values grab attention. Visualizations have the power to show the unexpected and to surprise the analysts, sometimes taking the analysis in unplanned but interesting directions. Interactive visualizations add even more value as they give tools to investigate the probable causes and to get additional information that could provide insight into data patterns. BRAVIZ implements mechanisms to identify individual points on every visualization and to look into each of them in detail. This is specially useful in data derived from image data, where anomalies can be related to characteristics of the images themselves. Looking at the images and spatial structures behind each measure improves interpretation and makes limitations visible. Additionally, displaying multiples types of spatial data in the same space helps researchers find or design better measures. For example, fiber bundles can be defined based on functional patterns, anatomy, local diffusion or by combining all of them.

%- Cleaner datasets
%- Time efficiency
% -- Do you spend more time looking at data now?
By highlighting anomalous points in the data-set, BRAVIZ contributes in cleaning the data-set. These anomalies can be investigated, and if mistakes are found (typing or data processing errors for example) they can be corrected. If this is not possible (for example images with heavy noise) the point can be excluded from further analyses. After each analysis iteration problems in the data-set will be less common, and analysts will feel more confident of the quality of the data and consequently of the results extracted from it. An additional effect is that the data-set becomes more valuable as time goes by, and therefore researchers are encouraged to come back to a data-set. An objective of the project was increasing the amount of information that can be extracted from each data-set, and that data collected for a certain study could outlive it, and be used in the future. Having a data-set that is not static, but that is continuously cleaned and enhanced with new and transformed data is a key step towards this objective.

%- Encourage exploratory Analysis
%- Share work across experts
%-- more communication
%- Freely explore, follow curiosity, leave room for surprises
It can be seen that BRAVIZ supports and encourages exploratory analyzes. By making the complete data-set available to every researcher it encourages interdisciplinary work. Having access to data from different fields permits experts explore data that was usually out of their reach. This causes an increase in curiosity and a desire to learn more, which fosters interaction with other experts. In addition, experts can use visualizations to support and explain their ideas to the rest of the group. Visualizations can also be shared and revisited independently by each expert. Each look at the data opens the door for finding the unexpected and new discoveries. 

%- based on the case studies and evidence

\section{Importance of the model}

%How the model helps
%- Support exploratory research workflow
%- More efficient use of developers and experts time
%- Improving data sharing
%- Improving collaboration

Chapters \ref{chap_kmc400} and \ref{chap_analysis} showed that BRAVIZ adapts to experts' workflow and that it is a valuable tool for exploratory analysis. Some of the applications that make up BRAVIZ are more popular among experts, and others are used only by a few. Nonetheless it has been shown that the applications complement each other and that they can be used together. For example, an expert radiologist can spend a considerable amount of time isolating an specific fibers bundle, and then the whole group can benefit from it.
This integration among applications is plausible because they are all based on the model described in chapter \ref{chap_model}. The model acknowledges the fact that there should be specific applications for specific tasks. Each application should be simple to use, but several applications can be used at different stages of complex tasks. The model encourages implementing features that will make the applications appropriate for integrated exploratory work flows. Punctually applications should report important user actions, they should communicate with other applications in the system and allow the user to save and restore the work at any point. 

Each BRAVIZ application is focused on an specific analysis task. Interfaces are explicitly designed and don't overwhelm the user with options. These small applications are straightforward to use, and thus users don't need extensive training. The user centered design method is continuously applied to detect and correct bottlenecks, and therefore increase efficiency in exploratory data analysis. 

%Because applications share a common framework, data sharing is direct. Several users of the system can therefore collaborate by using data or objects created by each other in any of the applications that make up the system. 
%- Adapt to different users / projects / scenarios
%- Understanding of the design space
%- Thinking of possible solutions
%- Faster coherent implementation, better maintainability
%- Development time, time to market
%- Solve/maintain common problems only once 
The problem space feature model (Figure \ref{fig_feature_problem}), which is the basis for each implementation, allows designers to effectively assess necessities of users and translate them into new application configurations. The feature model from Figure \ref{fig_feature_solution} lets designers discover reusable components and guarantees that that the new application will adapt to the  system. 

By reusing components across the different applications, difficult technical challenges can be addressed in detail in a single place, and the whole set of applications will benefit. Developers can trust these components, and focus their efforts on designing visualizations and interfaces, which are the components that make the most important difference for end users. Recall that the focus of the system is providing an efficient channel for users to work together with data and processing algorithms in a computer, following visual analytics principles. Therefore, third party processing algorithms should be reused, both for statistical and spatial data, whenever this is possible. These techniques reduce the developing cycles, and therefore allow more feedback rounds with end-users, which leads to designs closer to users needs.

The unique point for reading operations defined in the architecture lets developers adjust the complete system to different data formats and layouts by modifying a single module. This provides room for future improvements and applications in different projects. 

%- Based on Braviz

\section{Future Work}

\subsection{Visual Analytics}
%- Evaluation
This work presented evidence of the benefits of visual analytics systems in brain research, as well as the benefits of a domain model in the design and development of such systems. However, additional long term evaluations are needed in order to better assess the strengths and limitations of the proposal, such as complete long term in depth case studies \autocite{shneiderman_strategies_2006}. 

%- Better understanding of thought processes
	%-- Of experts workflow
	%-- Of collaborative research
	%-- Of extracting value from data
Visual analytics application rely on understanding human cognition and sense making. The current prototype integrates several of the best practices from experts in visual perception, but it can still benefit from a better understanding of the sense making processes that goes on in the experts' brain during exploratory research. As the understanding of these processes improve, it should be integrated into the model and used to further increase the efficiency of researchers.

%Challenges
%- Software
%-- Integrating more data
%-- More analyzes
%-- More statistics

\subsection{Software and Deployment}

The proposed software could also benefit from additional data types and corresponding visualization. In particular there is interest in integrating EEG and MEG data. In the past years analyzing fMRI data as networks has raised in popularity, specially by using resting state fMRI. Additionally there is a need for comparisons and analyzes involving groups of subjects, for example second level fMRI. Several users have also requested additional statistical tests, specifically non parametric tests, analysis of ordinal variables and multiple samples ANOVAs.  
%-- Additional tests
%-- Web,
%-- Cloud,
Web based visualizations are becoming the norm for data analysis. They have the advantage that of reaching directly a wide array of users on different devices. No installations are required, and there is no need for extensive modifications to support smart phones, tablets of different operating systems. However there are also benefits of native applications; as more control of resources (memory, cpu, network, etc), more options to interact (joysticks, wii-motes, oculus-rifts, haptic devices, etc), and possibly better performance (more optimizations are possible). Nevertheless it is worth exploring if it is convenient to migrate to a full web environment.

The current architecture has data, communications, processing and logging running on the user machine. This machine also runs a web server to  which other devices can communicate, but most calculations are still performed on the same machine. Another approach would be to move data and processing to a cloud , and let users run only the front-end (web or native) on their machines. This would be simpler for users as they would not require powerful machines and they could work from anywhere (home, office, travel). The trade-off is that a large investment would be required  to install and support the cloud. Storage and processing nodes on the cloud would also need to be adjusted depending on the number of users. One possibility would be implementing the system on a on-demand cloud server as Google cloud, Amazon or Open Shaft. In this case privacy and security issues would need to be carefully considered. 

%-- Massification
%-- Maintainance
%-- Model evolution
%-- Testing

\subsection{Massification and Maintenance}

At this moment, BRAVIZ is been actively used by a small community. A challenge for the future is making this user-base grow. This requires investments on communications, and support. The installation process and documentation require significant improvements. Robust mechanisms for evolving the system without breaking it for existing users should also be implemented. A crucial part of this is a good set of automatic tests. Ideally there would be several developers and maintainers, and this team would need organization and planning.

%- Data sharing, open science
% -policy
The value of solutions for exploratory analysis, such as BRAVIZ, will increase as more data becomes publicly available. This requires important changes in  culture and policy behind research. The current research funding system is highly competitive and does not encourage collaboration between teams. Also there are not incentives for exploring old data-sets, repeating experiments with new data or sharing data and methods with others. 

\subsection{Navigating through variables}

One of the main limitations users find when using BRAVIZ is understanding the meaning of variables and finding the correct variables for a question (see section \ref{sec_ana_results}). Currently variables can be searched by name, and additional information can be gathered from the description, variable type, and labels for nominal variables or range for numerical variables. However variable numbers are arbitrary. For example, the KMC data-set has variable names and descriptions in three different languages (English, Spanish and French). 

Also, while the naming conventions may be logical for members of the group, outsiders have a hard time getting use to them. In order to have datasets that can be exploited outside the group  that created it, it is necessary to design and implement standard mechanisms for naming and navigating variables. For example, by using controlled vocabularies and ontologies. There should also be additional mechanisms to search variables, for example by keywords and categories.

Finally, there should be \emph{recommendations} of potentially related variables. These should come from mining the data-set, mining available literature or mining analyzes histories.

%- Clinical?
\subsection{Other Applications}

Another questions that needs to be addressed in the future is if there are clinical applications of this proposal. If so, which ones are there, and what needs to be done in order to support them. One feature that has been recurrently requested by clinicians would be the possibility to integrate spatial data with neuro navigation systems for neuro-surgery. However these imposes several challenges (see for example \autocite{talos_diffusion_2003}) and because it imposes a  risk to the patient, such an application would require extensive testing and validation.
%- Other domains
%- Health, Cities, Industry, Economics
Finally, it should be explored if the proposal of multiple small applications with a large common base can be applied to visual analysis in other domains. Some examples could be analysis of transactions in banking systems, city planning, or public health.  