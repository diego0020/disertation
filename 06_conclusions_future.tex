%Answer research questions, close the gap, close the introduction
The previous chapters presented a model that allows to design and implement software tools that support exploratory analysis of data from large brain studies with heterogeneous data from multiple subjects. This model was used to implement a set of applications that supported the analysis of data from a large brain study which included several types of data derived from neuro-images as well as clinical, demographic and socio-economic data. Further tests with potential users showed the relevance of this type of tools in today's research. Tools like the ones proposed in this work can change the way in which information is extracted from data and help in the transition towards a more collaborative and open research. 

\section{Effects in research workflow}

%- Explicit contributions to experts (neuroscience, images, research)

%- Subject as a whole
%-- Integrate data
%-- In context

One of the benefits of the proposed approach is that it lets all researchers look at data from different fields. The effect of this is that there is an integrated view of each subject. Data is not considered in isolation, but in the context of other information from the subject, as well as in the context of the sample. This additional context allows experts to make better interpretations of the data they are focused in. Additionally, this allows researcher to raise questions and hypotheses relating the focus and context data. The benefits of integrating all study data in a single place were illustrated by researchers in the Kangaroo Mother Care study (see chapter \ref{chap_kmc400}) as well as physicians and researchers outside the foundation (see chapter \ref{chap_analysis}).\footnote{Help Cyril: Why is it important to look at subjects as a whole?} 

%- Deeper analysis
%-- Ask new questions
%-- complex questions
%- Extract more information from each data-set
By integrating several kinds of data in a single place, it becomes feasible to ask new questions, and explore the answers conveniently. The system goes further as the data-set can be continuously enhanced by adding new data, which increases the space to ask and explore. This new data is generated by transforming existing data or by manually generating new measures and objects derived from it. The final objective would be a system where technical difficulties never get in the way of the researcher's line of thought. By using the proposed system researchers have gone beyond the original questions and hypotheses of the project and are proposing several new questions on the same data-set. There are still moments where technical limitations get on the way of the analysis, but the system should continue growing in order to address these limitations and let researchers ask more complex questions, and make a better use of the collected data.

%- Better understanding through visualizations
In addition to integrating data from multiple domains the system focuses on presenting data visually. As evidenced in the visual analytics' literature, rich interactive graphics are the most efficient way of transferring information from a computer to the expert. In the domain of this study the benefits of visualizations over only numbers are also evident. Visualization of statistical results provide much more information than single numeric statistics. Plots can provide more accurate descriptions of the details of the relationships, as well as making unusual values grab attention. Visualizations have the power to show the unexpected and to surprise the analysts. Interactive visualizations add even more value as they just not show data behavior but gives tool to investigate the probable causes and to get additional information that could provide insight into this behavior. Braviz implements mechanisms to identify individual points on every visualization and look into each of them in detail. This is specially useful in data derived from image data, where anomalies can be explained by characteristics of the images themselves. By looking at the images and spatial structures behind each measure they can be better interpreted and their limitations can be made visible. Displaying all spatial data in the same space can also help researchers find or design better metrics for each question.

%- Cleaner datasets
%- Time efficiency
% -- Do you spend more time looking at data now?
By making anomalous points in the data-set evident, visualizations contribute to cleaning the data-set. These anomalies can be investigated, and if mistakes are found they can be corrected or if this is not possible the point can be excluded from the analysis. As the analysis continues mistakes in the data-set will be less common, and analysts can feel more confident of the quality of the data and the results that can be extracted from it. An additional effect is that the data-set becomes more valuable as time goes by, and therefore researchers are encouraged to come back to a data-set. An objective of the project was increasing the amount of information that can be extracted from each data-set, and that data collected for a certain study can outlive it and be used in the future. Having a data-set that is not static, but that is continuously cleaned and enhanced with new and transformed data is a key step towards this objective.

%- Encourage exploratory Analysis
%- Share work across experts
%-- more communication
%- Freely explore, follow curiosity, leave room for surprises
It can also be seen that the proposed system supports and encourages exploratory analyzes. By making the complete data-set available to every researcher it also encourages interdisciplinary work. Having access to data from different fields permits experts explore data that was usually out of their reach. This causes an increase in curiosity and a desire to learn more, which in turns promotes asking questions to other experts with more experience in the given data. Also, it is easier for each expert to use visualizations to support and explain their ideas to the rest of the group. This visualizations can be shared and revisited independently by each expert on their own data, which further supports teamwork. Each look at the data opens the door for finding the unexpected, which can lead to new discoveries. 

%- based on the case studies and evidence

\section{Importance of the model}

%How the model helps
%- Support exploratory research workflow
%- More efficient use of developers and experts time
%- Improving data sharing
%- Improving collaboration


%- Adapt to different users / projects / scenarios
%- Understanding of the design space
%- Thinking of possible solutions

%- Faster coherent implementation, better maintainability
%- Development time, time to market
%- Solve/maintain common problems only once 

%- Based on Braviz

\section{Future Work}

%- Evaluation
%- Better understanding of thought processes
	%-- Of experts workflow
	%-- Of collaborative research
	%-- Of extracting value from data
	%-- Of 

%Challenges
%- Software
%-- Model evolution
%-- Integrating more data
%-- Massification
%-- Maintainance
%-- Testing
%-- Web,
%-- Cloud,

%- Data sharing, open science

%- Clinical?
%- Other domains
%- Health, Cities, Industry, Economics